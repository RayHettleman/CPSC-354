\documentclass[11pt]{article}
\usepackage[margin=1in]{geometry}
\usepackage[utf8]{inputenc}
\usepackage[T1]{fontenc}

\title{The MIU System and the Impossibility of MIII}
\author{}
\date{}

\begin{document}
\maketitle

\section*{Goal}
The goal is to determine whether it is possible to derive \texttt{MIII} starting from \texttt{MI} using the rules of the MIU system.

\section*{Rules}
\textbf{Rule I:} If you possess a string whose last letter is \texttt{I}, you can add a \texttt{U} at the end.  

\textbf{Rule II:} Suppose you have \texttt{Mx}. Then you may add \texttt{Mxx} to your collection.  

\textbf{Rule III:} If \texttt{III} occurs in one of the strings in your collection, you may make a new string with \texttt{U} in place of \texttt{III}.  

\textbf{Rule IV:} If \texttt{UU} occurs inside one of your strings, you can drop it.  

\section*{Attempts}
\noindent
\begin{verbatim}
                            v—----------------------v
MI - MIU - MIUIU - MIUUIUU - MII - MIIU - MIIUU - MII *
                            MIIII - MUI - MUI - MUIIU
*THOUGHT* If possible, try to get MIII
\end{verbatim}

For example: starting with one \texttt{I}, Rule II can double it, giving two total \texttt{I}'s with a \texttt{U} in between. Because the only way for the number of \texttt{I}'s to grow is doubling, every time Rule III might apply, there will always be one leftover. This means it is never possible to reach exactly three \texttt{I}'s.

\section*{Conclusion}
None of the rules leads to the number of \texttt{I}'s being three.  

\begin{itemize}
    \item Rule I has no effect on the number of \texttt{I}'s.  
    \item Rule II can only double the \texttt{I}'s, which means they are always even after the first step. Getting exactly three \texttt{I}'s is therefore impossible.  
    \item Rule III is the important rule, because if we could get three \texttt{I}'s, this is how we would turn it into \texttt{MU}. But since we can’t reach three \texttt{I}'s, the rule never helps.  
    \item Rule IV has no effect on the number of \texttt{I}'s.  
\end{itemize}

Therefore, it is impossible to derive \texttt{MIII} from \texttt{MI}.

\end{document}
